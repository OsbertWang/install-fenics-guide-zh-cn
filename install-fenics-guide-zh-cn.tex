\documentclass[fontset=founder]{ctexrep}
\usepackage{accsupp}
\usepackage[margin=2.4cm]{geometry}
\usepackage{listings}
\usepackage{xcolor}
\usepackage{fontawesome5}
\usepackage[pdfpagelayout=SinglePage]{hyperref}
\usepackage[os=win,hyperrefcolorlinks]{menukeys}

\ctexset{
  section/format = \Large\bfseries\raggedright,
}

\lstset{
  backgroundcolor = \color{lightgray!30},
  keywordstyle    = \color{blue},
  stringstyle     = \color{brown},
  basicstyle      = {\small\ttfamily},
  breaklines      = true,
  tabsize         = 4,
  gobble          = 2,
  numbers         = left,
  numberstyle     = \tiny\emptyaccsupp,
  frame           = single,
  xleftmargin     = \ccwd,
  numbersep       = \ccwd,
  columns         = fullflexible,
%  emphstyle       = {\color{blue}\small\ttfamily},
%  emph            = {mkdir,rmdir,sudo,mount,umount,rm},
}

\newcommand\emptyaccsupp[1]{%
  \BeginAccSupp{ActualText={}}#1\EndAccSupp{}%
}

\renewmenumacro{\menu}[>]{angularmenus}
\renewmenumacro{\keys}[+]{shadowedroundedkeys}

\title{\bfseries 一份简短的关于 FEniCS 安装的介绍%
  \thanks{\url{https://github.com/OsbertWang/install-fenics-guide-zh-cn}}%
}
\author{王然%
  \thanks{\href{mailto:ranwang.osbert@outlook.com}%
    {\ttfamily ranwang.osbert@outlook.com}}%
}
\date{\today}

\begin{document}
  
\maketitle

\begin{abstract}
  
  本文记录了我安装 FEniCS 的经历.
  
  最开始,
  我将 FEniCS 安装在 WSL 的 Ubuntu 20.04 下.
  本人 Ubuntu 使用经验并不丰富,
  因此很可能会有记录错误的地方.
  另外 FEniCS 现已停止开发,
  其后续版本 FEniCSx 还未稳定.
  这无疑给用户带来了不安.
  
  2021年10月23日,
  我尝试安装 \href{https://www.docker.com/get-started}{Docker}.
  因此补充了 Docker 下安装 FEniCS 的步骤.
  
  而后的2021年10月28日,
  我正式将 FEniCSx 的安装步骤写入手册.
  根据当时的官方的建议,
  我只撰写了 Docker 下的安装步骤.

  2023年3月25日,
  在 Windows 11 下的 WSL 中安装了 Ubuntu 22.04,
  此次尝试安装 FEniCSx.
  
  希望本手册能够为用户打开方便.
\end{abstract}

\tableofcontents

\chapter{WSL 中安装 FEniCSx}

\section{安装 WSL}\label{sec:wsl.install}

在网上能够找到许多安装 WSL 的教程,
例如%
\href{https://learn.microsoft.com/zh-cn/windows/wsl/install}{微软官方文档}.
然而,
由于大陆网络无法自由访问备份于 \url{raw.githubusercontent.com} 的
\href{https://raw.githubusercontent.com/microsoft/WSL/master/distributions/DistributionInfo.json}{WSL 的 json} 文件,
这会导致使用
\begin{lstlisting}
  wsl --install
\end{lstlisting}
直接安装失败,
因此不得不采用另外的方案,
例如%
\href{https://learn.microsoft.com/zh-cn/windows/wsl/install-manual}{旧版手动安装步骤}%
或者修改 hosts.

具体而言,
先在一些 ip 查询网站找到 \url{raw.githubusercontent.com} 对应的 ip,
比如目前我找到的就是以下内容
\begin{lstlisting}
  185.199.111.133 raw.githubusercontent.com
\end{lstlisting}
接下来用管理员身份打开
\begin{lstlisting}
  C:\Windows\System32\drivers\etc\hosts
\end{lstlisting}
将查询到的 ip 内容添加到文件末尾.
这时再执行
\begin{lstlisting}
  wsl -l -o
\end{lstlisting}
就可以看到当前允许安装的 WSL 发行版
\begin{lstlisting}
  NAME                                   FRIENDLY NAME
  Ubuntu                                 Ubuntu
  Debian                                 Debian GNU/Linux
  kali-linux                             Kali Linux Rolling
  Ubuntu-18.04                           Ubuntu 18.04 LTS
  Ubuntu-20.04                           Ubuntu 20.04 LTS
  Ubuntu-22.04                           Ubuntu 22.04 LTS
  OracleLinux_8_5                        Oracle Linux 8.5
  OracleLinux_7_9                        Oracle Linux 7.9
  SUSE-Linux-Enterprise-Server-15-SP4    SUSE Linux Enterprise Server 15 SP4
  openSUSE-Leap-15.4                     openSUSE Leap 15.4
  openSUSE-Tumbleweed                    openSUSE Tumbleweed
\end{lstlisting}

% 注意到部分 WSL2 用户反馈,
% 启动时提示``参考的对象类型不支持尝试的操作'',
% 在网上有一种流行的解决方案是
% \begin{itemize}
%   \item 下载软件 \href{https://www.proxifier.com/tmp/Test20200228/NoLsp.exe}{NoLsp.exe};
%   \item Powershell 管理员权限运行 (注意 Nolsp.exe 的位置)
%   \begin{lstlisting}[gobble=4]
%     .\NoLsp.exe C:\windows\system32\wsl.exe  
%   \end{lstlisting}
%   \item 如过上述链接无法下载,
%   请访问\href{https://cowtransfer.com/s/626f2c73fc7a4f}{这里}下载.
% \end{itemize}

\section{选择编辑器}

目前 Windows 11 已经允许 WSL 可视化,
但我还是更偏向于使用
\href{https://code.visualstudio.com/}{VS Code},
在%
\href{https://learn.microsoft.com/zh-cn/windows/wsl/tutorials/wsl-vscode}{微软官方文档}%
中有详细的介绍.

\section{更换 Ubuntu 源}\label{sec:source}

从本节开始,
有关命令行的操作,
在未特殊声明的情况下,
全部为 WSL 终端中的操作.

鉴于中国大陆地区网络状况的特性,
强烈推荐用户安装 FEniCSx 前更改 Ubuntu 的源.
在这里,
我使用%
\href{https://mirrors.tuna.tsinghua.edu.cn/help/ubuntu/}{清华大学镜像}.
当然还有其他镜像,
例如%
\href{https://mirrors.ustc.edu.cn/help/ubuntu.html}{中国科技大学镜像}.

在终端中执行
\begin{lstlisting}[language = bash]
  sudo cp /etc/apt/sources.list /etc/apt/sources.list.bak
\end{lstlisting}
备份 \textsf{sources.list} 文件.
接下来执行
\begin{lstlisting}[language = bash]
  cp /etc/apt/sources.list ~/sources.list
\end{lstlisting}
复制一份 \textsf{sources.list} 到用户文件夹.
\begin{lstlisting}[language = bash]
  code ~/sources.list
\end{lstlisting}
打开文件,
将文件替换为以下内容
\begin{lstlisting}
  # 默认注释了源码镜像以提高 apt update 速度,如有需要可自行取消注释
  deb https://mirrors.tuna.tsinghua.edu.cn/ubuntu/ jammy main restricted universe   multiverse
  # deb-src https://mirrors.tuna.tsinghua.edu.cn/ubuntu/ jammy main restricted   universe multiverse
  deb https://mirrors.tuna.tsinghua.edu.cn/ubuntu/ jammy-updates main restricted   universe multiverse
  # deb-src https://mirrors.tuna.tsinghua.edu.cn/ubuntu/ jammy-updates main   restricted universe multiverse
  deb https://mirrors.tuna.tsinghua.edu.cn/ubuntu/ jammy-backports main   restricted universe multiverse
  # deb-src https://mirrors.tuna.tsinghua.edu.cn/ubuntu/ jammy-backports main   restricted universe multiverse
  
  # deb https://mirrors.tuna.tsinghua.edu.cn/ubuntu/ jammy-security main   restricted universe multiverse
  # # deb-src https://mirrors.tuna.tsinghua.edu.cn/ubuntu/ jammy-security main   restricted universe multiverse
  
  deb http://security.ubuntu.com/ubuntu/ jammy-security main restricted universe   multiverse
  # deb-src http://security.ubuntu.com/ubuntu/ jammy-security main restricted   universe multiverse
  
  # 预发布软件源,不建议启用
  # deb https://mirrors.tuna.tsinghua.edu.cn/ubuntu/ jammy-proposed main   restricted universe multiverse
  # # deb-src https://mirrors.tuna.tsinghua.edu.cn/ubuntu/ jammy-proposed main   restricted universe multiverse
\end{lstlisting}
然后执行
\begin{lstlisting}[language = bash]
  sudo cp ~/sources.list /etc/apt/sources.list
\end{lstlisting}
替换原始的 \textsf{sources.list} 文件.
之后执行
\begin{lstlisting}[language=bash]
  sudo apt update && sudo apt upgrade
\end{lstlisting}
换源并更新.
若更改错误,
可执行
\begin{lstlisting}[language=bash]
  sudo cp /etc/apt/sources.list.bak /etc/apt/sources.list
\end{lstlisting}
恢复文件.

\section{添加 PPA}

在终端中执行
\begin{lstlisting}[language = bash]
  sudo add-apt-repository ppa:fenics-packages/fenics
\end{lstlisting}
鉴于网络原因,
直接从 PPA 下载容易丢内容.
这里介绍中科大的\href{https://lug.ustc.edu.cn/wiki/mirrors/help/revproxy/}{反向镜像},
将 \textsf{fenics-packages-ubuntu-fenics-focal.list} 备份
\begin{lstlisting}[language = bash]
  sudo cp /etc/apt/sources.list.d/fenics-packages-ubuntu-fenics-focal.list /etc/apt/sources.list.d/fenics-packages-ubuntu-fenics-focal.list.bak
\end{lstlisting}
接下来复制到用户文件夹
\begin{lstlisting}[language = bash]
  cp /etc/apt/sources.list.d/fenics-packages-ubuntu-fenics-focal.list ~/fenics-packages-ubuntu-fenics-focal.list
\end{lstlisting}
打开文件
\begin{lstlisting}[language = bash]
  code ~/fenics-packages-ubuntu-fenics-focal.list
\end{lstlisting}
将文件改为
\begin{lstlisting}
  deb https://launchpad.proxy.ustclug.org/fenics-packages/fenics/ubuntu focal main
  # deb-src https://launchpad.proxy.ustclug.org/fenics-packages/fenics/ubuntu focal main
\end{lstlisting}
而后再替换原文件
\begin{lstlisting}[language = bash]
  sudo cp ~/fenics-packages-ubuntu-fenics-focal.list /etc/apt/sources.list.d/fenics-packages-ubuntu-fenics-focal.list
\end{lstlisting}
最后再更新
\begin{lstlisting}[language = bash]
  sudo apt update
\end{lstlisting}
即完成添加 PPA 的操作.

\section{安装 FEniCS}

在终端中执行
\begin{lstlisting}[language = bash]
  sudo apt install build-essential
\end{lstlisting}
为后续做准备.
接下来执行
\begin{lstlisting}[language = bash]
  sudo apt install fenics
\end{lstlisting}
即可开始安装.

\section{测试安装}

在终端中执行
\begin{lstlisting}[language = bash]
  python3 -c 'import fenics'
\end{lstlisting}
若没有任何报错提示,
则说明安装成功.

\section{目前可用的入门资料}

目前 FEniCS 的开发已经不再活跃,
团队投放更多精力在开发 FEniCSx,
因此 FEniCS 的入门资料也就缺乏维护.

目前在 FEniCS 官网上还可以找到
\href{https://fenicsproject.org/tutorial/}{FEniCS Tutorial},
但其中一些内容已不适用最新版本,
建议用户搭配 \href{https://github.com/hplgit/fenics-tutorial/issues}{Github issues}
来阅读书中的内容.
除教程外,
用户还可以阅读
\href{https://fenicsproject.org/olddocs/dolfin/latest/python/demos.html}{Demos}
来学习部分用法.

\chapter{Docker 中安装 FEniCS}

\section{安装 WSL2}\label{sec:install-wsl2}

对于 Windows 10 家庭版用户,
使用 Docker 前需要安装 WSL2.
具体步骤可以参考第 \ref{sec:wsl.install} 节.
非 Windows 10 家庭版用户可以越过此步骤.

\section{安装 Docker}\label{sec:install-docker}

在安装完 WSL2 后,
用户可以下载 \href{https://www.docker.com/get-started}{Docker 桌面版}.
安装并重启,
将开启 Docker 教学环节.
注意,
在目前的 Windows 版本中,
Docker 默认不能改变安装路径.
更加详细的内容可以看\href{https://docs.docker.com/desktop/windows/install/}{这里}.

\section{安装 FEniCS}

打开 \textsf{cmd} 或者 \textsf{PowerShell},
进入打算安装 FEniCS 的路径中.
接下来执行
\begin{lstlisting}[language=bash]
  docker pull quay.io/fenicsproject/stable:latest
\end{lstlisting}
等待系统执行完毕即可.

\section{运行 FEniCS 容器}

依旧在 \textsf{cmd} 或者 \textsf{PowerShell} 中,
执行
\begin{lstlisting}[language=bash]
  docker run -ti quay.io/fenicsproject/stable:latest
\end{lstlisting}
系统最终将返回类似
\begin{lstlisting}
  fenics@d66e6f16a673:~$
\end{lstlisting}
的结果,
这就表明用户已经运行了 FEniCS 的容器,
也就表明我们的安装已经成功.
这里的 \texttt{d66e6f16a673} 是此容器的 ID.

\section{共享文件夹}\label{sec:shared-folder}

上述步骤结束后,
我们已经能够使用 Docker 中的 FEniCS.
然而我们看不到其他文件夹内的文件,
也无法调用常用的编辑器、.
因此,
现在需要将已有文件夹共享到 Docker 容器中.

比如我有一个文件夹 \texttt{D:\textbackslash work-fenics},
里面已经保存了我已写好的文件.
接下来运行
\begin{lstlisting}[language=bash]
  docker run -ti -v D:/work-fenics:/home/fenics/shared --name fenics-container quay.io/fenicsproject/stable:latest
\end{lstlisting}
将开启新的容器,
并且给它命名为 \texttt{fenics-container}.
输入
\begin{lstlisting}[language=bash]
  ls
\end{lstlisting}
可以看到这里多了一个 \texttt{shared} 文件夹,
进入后发现 \texttt{D:\textbackslash work-fenics} 的内容均在其中.

\section{查询容器}

执行
\begin{lstlisting}
  docker ps
\end{lstlisting}
可以看到目前正在运行的容器的信息.
执行
\begin{lstlisting}
  docker ps -a
\end{lstlisting}
可以查询所有容器的信息.

\section{退出、停止、再进入甚至删除容器}\label{sec:handle-containder}

与一般的命令行操作一样,
在
\begin{lstlisting}
  fenics@d66e6f16a673:~$
\end{lstlisting}
中输入
\begin{lstlisting}
  exit
\end{lstlisting}
就退出了容器,

如果要停止依然启动的容器,
可以执行
\begin{lstlisting}
  docker stop d66e6f16a673
\end{lstlisting}
或者
\begin{lstlisting}
  docker stop fenics-container
\end{lstlisting}
停止已有的容器,
前者是容器的 ID,
后者是容器的名称.

想要再进入容器,
需要先启动容器,
只需要执行
\begin{lstlisting}
  docker start fenics-container
\end{lstlisting}
然后再执行
\begin{lstlisting}
  docker exec -ti -u fenics fenics-container /bin/bash -l
\end{lstlisting}
便可进入容器.

如果不再需要容器,
执行
\begin{lstlisting}
  docker rm fenics-container
\end{lstlisting}
可以直接删除它,

\section{编译文件}

我们假设在 \texttt{D:\textbackslash work-fenics} 中有个 \texttt{demo.py} 文件.
使用上述方法进入容器后,
可以进入 \texttt{shared} 文件夹
\begin{lstlisting}
  cd shared
\end{lstlisting}
然后执行
\begin{lstlisting}
  python3 demo.py
\end{lstlisting}
编译.

注意,
这种方法不允许用户使用 \texttt{matplotlib.pyplot.show()} 查看图形结果.
还是建议用户多使用 \href{https://www.paraview.org/}{Paraview} 做后处理.

\section{文献资料}

有关在 Docker 中使用 FEniCS,
目前比较权威的资料是 \href{https://fenics.readthedocs.io/projects/containers/en/latest/index.html}{FEniCS Containers Document}.
它里面包含了很多用法,
例如 \href{https://fenics.readthedocs.io/projects/containers/en/latest/work_flows.html#run-fenics-in-a-docker-container-like-an-application}{Run FEniCS in a Docker container like an application}.
有兴趣的用户还请认真阅读.

\chapter{FEniCSx 安装}

目前 FEniCS 团队已经专心维护 FEniCSx,
而不再继续开发 FEniCS,
因此本章将讲述如何安装 FEniCSx.

\section{安装 WSL2 和 Docker}

目前 FEniCS 团队推荐用户使用 Docker 安装 FEniCSx.
这部分内容请读者参考第 \ref{sec:install-wsl2} 节和第 \ref{sec:install-docker} 节.

\section{运行 FEniCSx 容器}

FEniCS 团队预构建了 FEniCSx 的 \href{https://hub.docker.com/r/dolfinx/dolfinx}{Docker 镜像}.
由于 DOLFINx 的 Docker 镜像托管在 Docker-hub.
用户可以直接访问该镜像
\begin{lstlisting}
  docker run -ti -v D:/work-fenicsx:/root/shared --name fenicsx-container dolfinx/dolfinx
\end{lstlisting}
以上代码中,
我将工作路径 \texttt{D:\textbackslash work-fenicsx} 设为共享路径,
容器名称为 \texttt{fenicsx-container}.
这些概念可以阅读第 \ref{sec:shared-folder} 节加以了解.

\section{退出、停止、再进入甚至删除容器}

这部分内容是 Docker 的操作,
可以阅读第 \ref{sec:handle-containder} 节加以了解.
注意再进入 \texttt{fenicsx-container} 容器,
执行
\begin{lstlisting}
  docker exec -ti fenicsx-container /bin/bash -l
\end{lstlisting}

\section{拉取镜像}

鉴于目前 FEniCSx 还处于开发状态,
因此很多时候需要拉取镜像来使用最新的版本.
在命令行中执行
\begin{lstlisting}
  docker pull dolfinx/dolfinx
\end{lstlisting}
系统就会拉取最新的内容到本地,
然后执行
\begin{lstlisting}
  docker image list
\end{lstlisting}
可以看到目前本地的全部镜像的信息.

\section{编译文件}

将 \href{https://docs.fenicsproject.org/dolfinx/main/python/_downloads/756b4617c0f8921ed14fbd158af0c99c/demo_poisson.py}{demo\_poisson.py} 下载至 \texttt{D:\textbackslash work-fenicsx},
进入容器,
在 \texttt{shared} 文件夹中执行
\begin{lstlisting}
  python3 demo_poisson.py
\end{lstlisting}
编译文件.

\section{安装 pyvista}

上面的例子运行后,
系统会出现
\begin{lstlisting}
  pyvista is required to visualise the solution
\end{lstlisting}
这是因为 \texttt{pyvista} 没有安装成功.
根据 \href{https://fenicsproject.discourse.group/t/how-to-use-pyvista-in-docker-for-windows-10-user/6921}{fenicsproject.discourse.group} 的回答,
简述安装 pyvista 的过程.

首先换源.
将 \texttt{sources.list} 复制到 \texttt{shared} 文件夹
\begin{lstlisting}
  cp /etc/apt/sources.list ~/shared/sources.list
\end{lstlisting}
然后在 \texttt{D:\textbackslash work-fenicsx} 中更改 \texttt{sources.list} 内容,
见第 \ref{sec:source} 节.
改后的文件复制回去
\begin{lstlisting}
  cp ~/shared/sources.list /etc/apt/sources.list
\end{lstlisting}

下面开始安装,
依次执行
\begin{lstlisting}
  export PYVISTA_OFF_SCREEN=true
  apt-get update
  apt-get install -y --no-install-recommends libgl1-mesa-dev xvfb
  pip3 install pyvista
\end{lstlisting}
即可安装完成.
如果论坛所说,
再次执行
\begin{lstlisting}
  python3 demo_poisson.py
\end{lstlisting}
将得到 \texttt{u.png} 文件.

\section{文献资料}

有关在 Docker 中使用 FEniCSx,
目前比较权威的资料是  \href{https://docs.fenicsproject.org/dolfinx/main/python/index.html}{DOLFINx documentation}.
此外还有  \href{https://jorgensd.github.io/dolfinx-tutorial/index.html}{FEniCSx 教程}.
有兴趣的用户还请认真阅读.

\end{document}