\documentclass[fontset=founder]{ctexart}
\usepackage{accsupp}
\usepackage[margin=2.4cm]{geometry}
\usepackage{listings}
\usepackage{xcolor}
\usepackage{fontawesome5}
\usepackage[pdfpagelayout=SinglePage]{hyperref}
\usepackage[os=win,hyperrefcolorlinks]{menukeys}

\ctexset{
  section/format = \Large\bfseries\raggedright,
}

\lstset{
  backgroundcolor = \color{lightgray!30},
  keywordstyle    = \color{blue},
  stringstyle     = \color{brown},
  basicstyle      = {\small\ttfamily},
  breaklines      = true,
  tabsize         = 4,
  gobble          = 2,
  numbers         = left,
  numberstyle     = \tiny\emptyaccsupp,
  frame           = single,
  xleftmargin     = \ccwd,
  numbersep       = \ccwd,
  columns         = fullflexible,
%  emphstyle       = {\color{blue}\small\ttfamily},
%  emph            = {mkdir,rmdir,sudo,mount,umount,rm},
}

\newcommand\emptyaccsupp[1]{%
  \BeginAccSupp{ActualText={}}#1\EndAccSupp{}%
}

\renewmenumacro{\menu}[>]{angularmenus}
\renewmenumacro{\keys}[+]{shadowedroundedkeys}

\title{\bfseries 一份简短的关于 FEniCS 安装的介绍%
  \thanks{\url{https://github.com/OsbertWang/install-fenics-guide-zh-cn}}%
}
\author{王然%
  \thanks{\href{mailto:ranwang.osbert@outlook.com}%
    {\ttfamily ranwang.osbert@outlook.com}}%
}
\date{\today}

\begin{document}
  
\maketitle

\begin{abstract}
  本文记录了我在 WSL 的 Ubuntu 20.04 下安装 FEniCS 的经历.
  本人 Ubuntu 使用经验并不丰富,
  因此很可能会有记录错误的地方,
  希望可以得到行家里手的批评指正.
  另外 FEniCS 现已停止开发,
  其后续版本 FEniCS-X 还未稳定.
  本文将在 FEniCS-X 发布稳定版本后再更新相关内容.
\end{abstract}

\section{安装 WSL}

网上已经有很多安装 WSL 的教程,
例如%
\href{https://docs.microsoft.com/zh-cn/windows/wsl/install-win10}{微软的文档}.
注意到部分 WSL2 用户反馈,
启动时提示``参考的对象类型不支持尝试的操作'',
在网上有一种流行的解决方案是
\begin{itemize}
  \item 下载软件 \href{https://www.proxifier.com/tmp/Test20200228/NoLsp.exe}{NoLsp.exe};
  \item Powershell 管理员权限运行 (注意 Nolsp.exe 的位置)
  \begin{lstlisting}[gobble=4]
    .\NoLsp.exe C:\windows\system32\wsl.exe  
  \end{lstlisting}
\end{itemize}
若有更好的解决方案,
还请告知于我.

\section{选择编辑器}

WSL 是完全使用命令行的子系统,
默认可以使用 vim,
但是 vim 的用法和 Windows 上很多编辑器不同,
因此无疑对用户造成了困难.

在这里,
我使用
\href{https://code.visualstudio.com/}{VS Code}
这个编辑器.
注意安装后要检查环境变量.
在 \textsf{cmd} 中执行
\begin{lstlisting}
  echo %path:;=&echo.%
\end{lstlisting}
可以列出目前可用的环境变量,
如果结果中包含
\begin{lstlisting}
  ...\Microsoft VS Code\bin\
\end{lstlisting}
就说明我们安装成功,
否则我们需要手动将其添加到环境变量中.
在 Windows 10 内安装并打开它,
使用快捷键 \keys{Ctrl + Shift + X} 打开 \textsf{Extensions (扩展)},
找到 \textsf{Remote - WSL} 并安装.
这样便可以直接在 WSL 中打开文档.

另外 VS Code 还有简体中文翻译扩展 \textsf{Chinese (Simplified) Language Pack for Visual Studio Code},
一定程度上避免了一些用户英语阅读不畅的问题.

\section{更换 Ubuntu 源}

从本节开始,
有关命令行的操作,
在未特殊声明的情况下,
全部为 WSL 终端中的操作.

鉴于中国大陆地区网络状况的特性,
强烈推荐用户安装 FEniCS 前更改 Ubuntu 的源.
在这里,
我使用%
\href{https://mirrors.tuna.tsinghua.edu.cn/help/ubuntu/}{清华大学镜像}.
当然还有其他镜像,
例如%
\href{https://mirrors.ustc.edu.cn/help/ubuntu.html}{中国科技大学镜像}.

在终端中执行
\begin{lstlisting}[language = bash]
  sudo cp /etc/apt/sources.list /etc/apt/sources.list.bak
\end{lstlisting}
备份 \textsf{sources.list} 文件.
接下来执行
\begin{lstlisting}[language = bash]
  cp /etc/apt/sources.list ~/sources.list
\end{lstlisting}
复制一份 \textsf{sources.list} 到用户文件夹.
\begin{lstlisting}[language = bash]
  code ~/sources.list
\end{lstlisting}
打开文件,
将文件替换为以下内容
\begin{lstlisting}
  # 默认注释了源码镜像以提高 apt update 速度, 如有需要可自行取消注释
  deb https://mirrors.tuna.tsinghua.edu.cn/ubuntu/ focal main restricted universe multiverse
  # deb-src https://mirrors.tuna.tsinghua.edu.cn/ubuntu/ focal main restricted universe multiverse
  deb https://mirrors.tuna.tsinghua.edu.cn/ubuntu/ focal-updates main restricted universe multiverse
  # deb-src https://mirrors.tuna.tsinghua.edu.cn/ubuntu/ focal-updates main restricted universe multiverse
  deb https://mirrors.tuna.tsinghua.edu.cn/ubuntu/ focal-backports main restricted universe multiverse
  # deb-src https://mirrors.tuna.tsinghua.edu.cn/ubuntu/ focal-backports main restricted universe multiverse
  deb https://mirrors.tuna.tsinghua.edu.cn/ubuntu/ focal-security main restricted universe multiverse
  # deb-src https://mirrors.tuna.tsinghua.edu.cn/ubuntu/ focal-security main restricted universe multiverse
  
  # 预发布软件源, 不建议启用
  # deb https://mirrors.tuna.tsinghua.edu.cn/ubuntu/ focal-proposed main restricted universe multiverse
  # deb-src https://mirrors.tuna.tsinghua.edu.cn/ubuntu/ focal-proposed main restricted universe multiverse
\end{lstlisting}
然后执行
\begin{lstlisting}[language = bash]
  sudo cp ~/sources.list /etc/apt/sources.list
\end{lstlisting}
替换原始的 \textsf{sources.list} 文件.
之后执行
\begin{lstlisting}[language=bash]
  sudo apt update && sudo apt upgrade
\end{lstlisting}
换源并更新.
若更改错误,
可执行
\begin{lstlisting}[language=bash]
  sudo cp /etc/apt/sources.list.bak /etc/apt/sources.list
\end{lstlisting}
恢复文件.

\section{添加 PPA}

在终端中执行
\begin{lstlisting}[language = bash]
  sudo add-apt-repository ppa:fenics-packages/fenics
\end{lstlisting}
鉴于网络原因,
直接从 PPA 下载容易丢内容.
这里介绍中科大的\href{https://lug.ustc.edu.cn/wiki/mirrors/help/revproxy/}{反向镜像},
将 \textsf{fenics-packages-ubuntu-fenics-focal.list} 备份
\begin{lstlisting}[language = bash]
  sudo cp /etc/apt/sources.list.d/fenics-packages-ubuntu-fenics-focal.list /etc/apt/sources.list.d/fenics-packages-ubuntu-fenics-focal.list.bak
\end{lstlisting}
接下来复制到用户文件夹
\begin{lstlisting}[language = bash]
  cp /etc/apt/sources.list.d/fenics-packages-ubuntu-fenics-focal.list ~/fenics-packages-ubuntu-fenics-focal.list
\end{lstlisting}
打开文件
\begin{lstlisting}[language = bash]
  code ~/fenics-packages-ubuntu-fenics-focal.list
\end{lstlisting}
将文件改为
\begin{lstlisting}
  deb https://launchpad.proxy.ustclug.org/fenics-packages/fenics/ubuntu focal main
  # deb-src https://launchpad.proxy.ustclug.org/fenics-packages/fenics/ubuntu focal main
\end{lstlisting}
而后再替换原文件
\begin{lstlisting}[language = bash]
  sudo cp ~/fenics-packages-ubuntu-fenics-focal.list /etc/apt/sources.list.d/fenics-packages-ubuntu-fenics-focal.list
\end{lstlisting}
最后再更新
\begin{lstlisting}[language = bash]
  sudo apt update
\end{lstlisting}
即完成添加 PPA 的操作.

\section{安装 FEniCS}

在终端中执行
\begin{lstlisting}[language = bash]
  sudo apt install build-essential
\end{lstlisting}
为后续做准备.
接下来执行
\begin{lstlisting}[language = bash]
  sudo apt install fenics
\end{lstlisting}
即可开始安装.

\section{测试安装}

在终端中执行
\begin{lstlisting}[language = bash]
  python3 -c 'import fenics'
\end{lstlisting}
若没有任何报错提示,
则说明安装成功.

\section{目前可用的入门资料}

目前 FEniCS 的开发已经不再活跃,
团队投放更多精力在开发 FEniCS-X,
因此 FEniCS 的入门资料也就缺乏维护.

目前在 FEniCS 官网上还可以找到
\href{https://fenicsproject.org/tutorial/}{FEniCS Tutorial},
但其中一些内容已不适用最新版本,
建议用户搭配 \href{https://github.com/hplgit/fenics-tutorial/issues}{Github issues}
来阅读书中的内容.
除教程外,
用户还可以阅读
\href{https://fenicsproject.org/olddocs/dolfin/latest/python/demos.html}{Demos}
来学习部分用法.

\end{document}
